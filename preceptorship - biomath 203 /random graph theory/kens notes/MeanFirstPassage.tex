\documentclass[11pt]{article}
\usepackage{amsfonts,amsmath,amssymb,amsthm}
\usepackage{amsbsy, amsmath,amsfonts}
\usepackage{graphicx,url}
\newtheorem{proposition}{Proposition}[section]
\newtheorem{example}{Example}[section]
\newcommand{\svskip}{\vspace{1.75mm}}
\newcommand{\mvskip}{\vspace{.25in}}
\newcommand{\lvskip}{\vspace{.5in}}
\def\E{\mathop{\rm E\,\!}\nolimits}
\def\Var{\mathop{\rm Var}\nolimits}
\def\Cov{\mathop{\rm Cov}\nolimits}
\def\den{\mathop{\rm den}\nolimits}
\def\midd{\mathop{\,|\,}\nolimits}
\def\sgn{\mathop{\rm sgn}\nolimits}
\def\vec{\mathop{\rm vec}\nolimits}
\def\sinc{\mathop{\rm sinc}\nolimits}
\def\curl{\mathop{\rm curl}\nolimits}
\def\div{\mathop{\rm div}\nolimits}
\def\tr{\mathop{\rm tr}\nolimits}
\def\len{\mathop{\rm len}\nolimits}
\def\diag{\mathop{\rm diag}\nolimits}
\def\amp{\mathop{\;\:}\nolimits}
\newcommand{\ba}{\boldsymbol{a}}
\newcommand{\bb}{\boldsymbol{b}}
\newcommand{\bc}{\boldsymbol{c}}
\newcommand{\bd}{\boldsymbol{d}}
\newcommand{\be}{\boldsymbol{e}}
\newcommand{\bff}{\boldsymbol{f}}
\newcommand{\bg}{\boldsymbol{g}}
\newcommand{\bh}{\boldsymbol{h}}
\newcommand{\bi}{\boldsymbol{i}}
\newcommand{\bj}{\boldsymbol{j}}
\newcommand{\bk}{\boldsymbol{k}}
\newcommand{\bl}{\boldsymbol{l}}
\newcommand{\bm}{\boldsymbol{m}}
\newcommand{\bn}{\boldsymbol{n}}
\newcommand{\bo}{\boldsymbol{o}}
\newcommand{\bp}{\boldsymbol{p}}
\newcommand{\bq}{\boldsymbol{q}}
\newcommand{\br}{\boldsymbol{r}}
\newcommand{\bs}{\boldsymbol{s}}
\newcommand{\bt}{\boldsymbol{t}}
\newcommand{\bu}{\boldsymbol{u}}
\newcommand{\bv}{\boldsymbol{v}}
\newcommand{\bw}{\boldsymbol{w}}
\newcommand{\bx}{\boldsymbol{x}}
\newcommand{\by}{\boldsymbol{y}}
\newcommand{\bz}{\boldsymbol{z}}
\newcommand{\bA}{\boldsymbol{A}}
\newcommand{\bB}{\boldsymbol{B}}
\newcommand{\bC}{\boldsymbol{C}}
\newcommand{\bD}{\boldsymbol{D}}
\newcommand{\bE}{\boldsymbol{E}}
\newcommand{\bF}{\boldsymbol{F}}
\newcommand{\bG}{\boldsymbol{G}}
\newcommand{\bH}{\boldsymbol{H}}
\newcommand{\bI}{\boldsymbol{I}}
\newcommand{\bJ}{\boldsymbol{J}}
\newcommand{\bK}{\boldsymbol{K}}
\newcommand{\bL}{\boldsymbol{L}}
\newcommand{\bM}{\boldsymbol{M}}
\newcommand{\bN}{\boldsymbol{N}}
\newcommand{\bO}{\boldsymbol{O}}
\newcommand{\bP}{\boldsymbol{P}}
\newcommand{\bQ}{\boldsymbol{Q}}
\newcommand{\bR}{\boldsymbol{R}}
\newcommand{\bS}{\boldsymbol{S}}
\newcommand{\bT}{\boldsymbol{T}}
\newcommand{\bU}{\boldsymbol{U}}
\newcommand{\bV}{\boldsymbol{V}}
\newcommand{\bW}{\boldsymbol{W}}
\newcommand{\bX}{\boldsymbol{X}}
\newcommand{\bY}{\boldsymbol{Y}}
\newcommand{\bZ}{\boldsymbol{Z}}
\newcommand{\balpha}{\boldsymbol{\alpha}}
\newcommand{\bbeta}{\boldsymbol{\beta}}
\newcommand{\bgamma}{\boldsymbol{\gamma}}
\newcommand{\bdelta}{\boldsymbol{\delta}}
\newcommand{\bepsilon}{\boldsymbol{\epsilon}}
\newcommand{\blambda}{\boldsymbol{\lambda}}
\newcommand{\bmu}{\boldsymbol{\mu}}
\newcommand{\bnu}{\boldsymbol{\nu}}
\newcommand{\bphi}{\boldsymbol{\phi}}
\newcommand{\bpi}{\boldsymbol{\pi}}
\newcommand{\bsigma}{\boldsymbol{\sigma}}
\newcommand{\btheta}{\boldsymbol{\theta}}
\newcommand{\bomega}{\boldsymbol{\omega}}
\newcommand{\bxi}{\boldsymbol{\xi}}
\newcommand{\bGamma}{\boldsymbol{\Gamma}}
\newcommand{\bDelta}{\boldsymbol{\Delta}}
\newcommand{\bTheta}{\boldsymbol{\Theta}}
\newcommand{\bLambda}{\boldsymbol{\Lambda}}
\newcommand{\bXi}{\boldsymbol{\Xi}}
\newcommand{\bPi}{\boldsymbol{\Pi}}
\newcommand{\bSigma}{\boldsymbol{\Sigma}}
\newcommand{\bUpsilon}{\boldsymbol{\Upsilon}}
\newcommand{\bPhi}{\boldsymbol{\Phi}}
\newcommand{\bPsi}{\boldsymbol{\Psi}}
\newcommand{\bOmega}{\boldsymbol{\Omega}}

\pagenumbering{gobble}

\title{Computation of Mean First Passage Times in a Markov Chain}
\author{Kenneth Lange
\\
\\
\\
Departments of Computational Medicine, \\
Human Genetics, and Statistics \\
University of California \\
Los Angeles, CA 90095\\
Phone: 310-206-8076 \\
E-mail klange@ucla.edu \\
\\
\\
\\
\\
Research supported in part by USPHS grants GM53275 and HG006139.}

\begin{document}
\maketitle
%\newpage
%\begin{abstract}

\noindent  \\ \\
\lvskip
%\noindent {\bf Key Words:} 

%\end{abstract}

\newpage
\pagenumbering{arabic}

\baselineskip=20pt

\section*{\center Introduction}

Let $\bP=(p_{ij})$ be the transition probability matrix of a finite
state Markov chain. The first passage time $T_{ij}$ is the number
number of steps it takes to reach $j$ starting from $i$. We will
assume all $p_{ii} =0$ and set $T_{ii}=0$ in contrast to the standard first passage time formulation. Let $m_{ij}=\E(T_{ij})$ be the expected value of $T_{ij}$ and collect the $m_{ij}$
into a matrix $\bM$. One can calculate the $m_{ij}$ recursively via
the system of equations
\begin{eqnarray*}
m_{ij} & = & 1 + \sum_{k}p_{ik}m_{kj}
\end{eqnarray*}
for all $j\ne i$. It is natural to solve this equation iteratively via the recurrence
\begin{eqnarray}
\bM^{(n+1)} & = & \bP\bM^{(n)} + {\bf 1} {\bf 1}^t \label{mean_recur}\\
\diag(\bM^{(n+1}) & = & {\bf 0} \nonumber 
\end{eqnarray}
starting from $\bM^{(0)}={\bf 0}$. This recurrence has several
advantages: (a) it exploits fast matrix times matrix multiplication,
(b) it correctly maintains the diagonal entries, (c) it is monotonic in the sense that $\bM^{(n+1)} \ge \bM^{(n)}$ for all $n$, and (d) it converges to the minimal solution of the equations. Monotonicity follows by induction from the form of the updates and the
obvious condition $\bM^{(1)} \ge \bM^{(0)}$. It is also clear by induction that if $\bM$ solves the system of equations, then $\bM \ge \bM^{(n)}$ for all $n$. In view of this bound, the monotonic sequence 
$\bM^{(n)}$ converges to a limit $\bM^{(\infty)} \le \bM$.

\begin{proposition}
The iterate $m_{ij}^{(n)}$ equals $\E(T_{ij}1_{\{T_{ij}\le n\}})$
and converges to $\E(T_{ij})$ whenever the latter is finite.
\end{proposition}
\begin{proof}
This identification is clearly true for $n=0$. Suppose $n>0$ and $X_1$ denotes the state of the chain after one step starting at $X_0=i$. Then 
\begin{eqnarray*}
m_{ij}^{(n+1)} & = & \E(T_{ij}1_{\{T_{ij}\le n+1\}}) \\
& = & \sum_{k} p_{ik} 
\E(T_{ij}1_{\{T_{ij}\le n+1\}} \mid X_1 = k) \\
& = & \sum_{k} p_{ik} 
\E\Big[(T_{kj}+1)1_{\{T_{kj}+1\le n+1\}}\Big] \\
& = & 1+ \sum_{k} p_{ik}\E(T_{kj}1_{\{T_{kj}\le n\}})\\
& = & 1+ \sum_{k} p_{ik}m_{kj}^{(n)}
\end{eqnarray*}
proves our first claim. The monotone convergence theorem implies that 
$m_{ij}^{(n)}$ converges to $\E(T_{ij})$ whenever the later is finite.
\end{proof} 

To understand the rate of convergence of the iteration scheme (\ref{mean_recur}),
we first observe that it operates column by column on $\bM$. Thus, we can isolate each
column and analyze its convergence rate separately. With out loss of generality, we choose column one. For the sake of convenience, we now decompose $\bP$ into the block matrix 
\begin{eqnarray*}
\bP & = & \begin{pmatrix} 0 & \ba \\
\bb & \bQ \end{pmatrix}.
\end{eqnarray*}
If we let $\bm_1$ denote the first column of $\bM$ minus its top
entry, then our recurrence for $\bm_1$ takes the form
\begin{eqnarray}
\bm_1^{(n+1)} & = & {\bf 1} + \bQ \bm_1^{(n)} \label{col_recur}
\end{eqnarray} 
with $\bm_1 = {\bf 0}$.
\begin{proposition}
Suppose $\bP$ is irreducible. In the notation just established, 
the map $f(\bm) = {\bf 1} + \bQ \bm$ is a contraction for some
norm with contraction constant $\rho <1$. Hence, the iterates (\ref{col_recur})
converge at a linear rate to the corresponding vector of mean first passage times. 
\end{proposition}
\begin{proof} The matrix $\bI-\bQ$ is diagonally dominant because the row sums
of $\bP$ equal 1. At least one row sum of $\bQ$ is strictly less than 1 since
otherwise $\bP$ is reducible. Hence, $\bI-\bQ$ is irreducibly diagonally dominant, and
Theorem 6.2.6 of \cite{ortega90} implies that $\bI-\bQ$ is nonsingular. Theorem
6.2.9 of \cite{ortega90} in turn implies that $\bQ$ has dominant eigenvalue $\omega<1$.
For given $\epsilon>0$, Theorem 1.3.6 of \cite{ortega90} implies that there exists a 
vector norm $\|\bx\|$ with induced matrix norm satisfying $\|\bQ\| \le \omega+\epsilon$. Taking $\rho = \omega + \epsilon < 1$ gives
\begin{eqnarray*}
\|f(\bm)-f(\bn)\| & \le & \|\bQ\| \cdot \|\bm-\bn\| \amp = \amp \rho\|\bm-\bn\|. 
\end{eqnarray*}
It follows that $f(\bm)$ is a contraction.  Finally, an appeal to the classical contraction theorem mapping theorem (Theorem 8.2.2 of \cite{ortega90}) yields convergence at the linear rate $\rho$. 
\end{proof}


\begin{thebibliography}{99}
\bibitem{langeapplied2010}
Lange, K (2010) {\it Applied Probability}, 2nd ed. Springer
\bibitem{ortega90}
Ortega JM (1990) {\it Numerical Analysis: A Second Course}. SIAM
\end{thebibliography}

\end{document}
